\section{Related Work}
Es gibt verschiedene Metamodelle wie beispielsweise Ecore, \textbf{G}eneric \textbf{M}odeling \textbf{E}nvironment (GME) \cite{ledeczi2001generic} und WebGME \cite{maroti2014online}. Jedes dieser Modelle speichert Kanten als ein Teil der dazugehörigen Knoten. Im beispielhaften Falle eines UML-Klassendiagramms bedeutet das, dass Referenzen in den jeweiligen Klassen gespeichert werden. Diese Methode führt dazu, dass es zu sogenannten Memory Heaps kommt. \\
\textit{Ein Memory Heap ist ein problematisches Speicherphänomen, bei dem der Speicher durch seine Struktur und seine abhängigen Elemente belegt wird. Das bedeutet, dass keine Teile des Speichers freigegeben werden können, obwohl eigentlich nur ein kleiner Teil dieses Speichers tatsächlich genutzt wird. \url{https://pubs.vmware.com/vfabric52/index.jsp?topic=/com.vmware.vfabric.em4j.1.2/em4j/conf-heap-management.html}} \\
Im Bezug auf das Klassendiagramm bedeutet dies, dass bei einer Klasse mit vielen Referenzen jede einzelne Referenz im Speicher vorgehalten werden muss. Das hier aufgeführte Phänomen wird an anderer Stelle aufgegriffen und genauer erläutert \cite{scheidgen2013reference}. Ein Ansatz dieses Problem zu umgehen ist Knoten und Referenzen bzw. Kanten in eigenständig zu speichern. \\
Das Problem das Modelle nicht partiell geladen werden können ist dem XML-Format geschuldet. Hier wurden an anderer Stelle schon Grundvoraussetzungen diskutiert und zu einem Kriterienkatalog zusammengefasst, um Modelle teilweise zu laden \cite{kolovos2013research}. Das JSON-Format erfüllt nicht alle Kriterien dieses Katalogs, bietet aber eine erste Grundlage von der aus weiter entwickelt werden kann.