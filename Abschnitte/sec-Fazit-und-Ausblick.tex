\section{Fazit und Ausblick}
Das hier vorgestellte Metamodell MoDiGen ist ein neuer Ansatz im Bereich DSML und biete neue Möglichkeiten bei einem geringeren Speicherplatzbedarf. Die Art der Speicherung von Kanten als eigenständige Objekte ermöglicht neben der Speicherplatzreduzierung einen \textit{einfachen/direkten} programmatischen Zugriff. Dies lässt aber auch zu, dass Kanten verwaisen können, indem ihre dazugehörigen Knoten gelöscht werden. Deshalb muss der jeweilige Modellierer angeben, ob eine Kante automatisch mit den dazugehörigen Knoten gelöscht wird. In Ecore hingegen werden werden die Kanten eines Knoten direkt mitgelöscht.\\
Das JSON-Format benötigt weniger Speicherplatz als XML-Format und erlaubt eine Darstellung auf Webseiten. Darüber hinaus können weitere Anpassungen und Speicherplatzoptimierungen in Javascirpt durchgeführt werden, um die Skalierbarkeit zu verbessern. \\
Dies sind die Neuerungen und Vorteile von MoDiGen gegenüber  anderen DSML-Tools. Neben den genannten Vorteilen gibt es noch einen weiteren Unterschied zu Ecore. In Ecore können zusätzlich zu Knoten und Kanten Operationen mit \textbf{EO}perationzu definiert werden. Dies ist aber nur ein Platzhalter der im Eclipse Modeling Frameworks implementiert werden kann. MoDiGen bietet diese Möglichkeit nicht. \\