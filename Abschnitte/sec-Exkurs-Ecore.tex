\section{Das Ecore-Modell}
In diesem Absatz wird der relevante Ausschnitt aus der Architektur des ECore-Modells n\"aher beschrieben, da es im Abschnitt \ref{sec:evaluation} zum Vergleich herangezogen wird.

\subsection{Architektur}
Der zugrundeliegende Aufbau des ECore-Modells bildet weitaus mehr modellierungs Möglichkeiten ab, als der Aufbau von MoDiGen (siehe dazu \cite{eclipse_ecore}). Dazu gehören u.a. Pakete (EPackage), Operationen mit Parametern (EOperation, EParameter) und EFactory durch die sich Instanzerzeugung oder Konvertierungen modellieren lassen. Diese Bestandteile des ECore-Modells werden bei der folgenden Betrachtung ausgelassen, da sie keinen Beitrag zum Verständnis der Evaluierung in Abschnitt \ref{sec:evaluation} leisten.\\
\\
Eine Klasse \textbf{EClass} hat einen Namen und besitzt Attribute \textbf{EAttribute} und Referenzen \textbf{EReference}. Die Referenz ist über eine Komposition modelliert und kann nicht ohne eine zugeordnete Klasse existieren. Eine abstrakte Klasse \textbf{ETypedElement} die von \textbf{EStructuralFeature} geerbt wird enth\"alt Attribute wie \textit{upper-} und \textit{lowerBound}, durch die sich Multiplizitäten festlegen lassen. Neben \textbf{EReference} erbt auch \textbf{EAttribute} von \textbf{EStructuralFeature}. Der Datentyp eines Attributes wird durch eine Assoziation zu \textbf{EDataType} festgelegt. Ecore definiert dabei 21 Java Datentypen, wie u.a. \textbf{EBoolean}, \textbf{EByte} und \textbf{EChar} sowie 10 zus\"atzliche (u.a. \textbf{EDate} und \textbf{EEList}). Das Modell ist unter \cite{eclipse_core} vollst\"andig einzusehen.\\
\\
Das Listing \ref{ecore-model} zeigt wie die Klasse \textbf{Female} aus Abbildung \ref{fig:Family-Tree-Model} als ECore-Modell in XML abgespeichert wird. Die Referenzen \textit{isWife} und \textit{isMother} werden durch \textbf{eStructuralFeatures} mit dem XML-Typ \textit{ecore:EReference} als Kindelemente beschrieben. Die entsprechenden Attribute wie \textit{eType} und \textit{upperBound} sind als XML-Attribute von \textbf{eStructuralFeatures} vermerkt.

\begin{lstlisting}[caption=Klasse Female als Ecore in XML, label=lst:ecore-modell, language=xml, basicstyle=\small, tabsize=4,frame=single, showstringspaces=false, numbers=left, keywordstyle=\bfseries, breaklines=true]
[...]
  <eClassifiers xsi:type="ecore:EClass" name="Female" eSuperTypes="#//Person">
    <eStructuralFeatures xsi:type="ecore:EReference" name="isWife" eType="#//Male"/>
    <eStructuralFeatures xsi:type="ecore:EReference" name="isMother" upperBound="-1" eType="#//Person"/>
  </eClassifiers>
[...]
\end{lstlisting}
