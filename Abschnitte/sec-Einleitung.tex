\section{Einleitung}
Das Paper \textit{\glqq Approach to Define Highly Scalable Metamodels Based on JSON\grqq}\cite{gerhart2015approach} wurde im Rahmen der BigMDE 2015 veröffentlicht und beschreibt die Gestaltung eines neuen Ansatzes zur Speicherung von \textbf{D}omain-\textbf{S}pecific \textbf{M}odeling \textbf{L}anguages (DSMLs). \\
DSMLs werden in der Softwareentwicklung eingesetzt, um aus verschiedenen Diagrammen Quelltext zu generieren. Alternativ können aber auch aus textuellen Inhalten Diagramme generiert werden, um sie so übersichtlicher und verständlicher zu machen. Hiermit gemeint ist beispielsweise die Generierung eines Klassendiagramms zu einem dazugehörigen Java-Programm. Dies kann entweder durch den Quellcode an sich geschehen oder durch spezielle \texttt{@Annotationen} die im Quellcode gesetzt werden müssen \cite{france2005domain}. \\
Zur Speicherung dieser Diagramme wurde in der Vergangenheit oft auf XML zurückgegriffen. Für Modelle die mit wenigen Knoten und Kanten auskommen ist dies unproblematisch. In großen Modellen ist die Anzahl der beiden Elemente aber unter Umständen so hoch, dass die angelegten XML-Dateien mehrere Gigabyte Speicher beanspruchen.\\ 
Neben dem großen Speicherbedarf kommt hier ein Nachteil der dem XML-Format geschuldet ist zum Tragen. Modelle können nur als Ganzes geladen bzw. ausgelesen werden. Ein streaming ist somit nicht möglich und die Betrachtung von Teilen des Modells erfordert das Laden des Gesamtmodells. Somit sind diese Modelle nicht skalierbar im Bezug auf ihren Speicherbedarf. Diese Limitierungen der Skalierbarkeit können durch das JSON-Format verbessert und zum Teil sogar aufgehoben werden. Beispielsweise kann der Speicherbedarf durch einen geringeren Overhead im Quellcode reduziert werden und die Modelle können über sogenannte \textit{Identifier} partiell geladen werden. \\
Um die zuvor getroffenen Aussagen zu untermauern und die Vorteile des JSON-Formats faktisch darzustellen wird im Folgenden eine JSON-DSML-Implementierung beschrieben und in einem anschließenden Benchmark mit Ecore verglichen. \\
Die Gliederung dieser Zusammenfassung orientiert sich am Originalpaper \cite{gerhart2015approach}. An einigen Stellen in dieser Zusammenfassung befinden sich Einschübe, um das im Original vorgestellte Metamodell MoDiGen genauer zu beschreiben oder um Hintergründe für Design-Entscheidungen verständlicher und durchschaubarer zu machen. 

\todo{Überprüfen und ggf. erweitern}{Peet}

%Dabei geht es nicht um eine gut lesbare und verständliche Übersicht, sondern um die Generierung der Diagramme. Hierbei gibt es verschiedene Ansätze zur Generierung.
