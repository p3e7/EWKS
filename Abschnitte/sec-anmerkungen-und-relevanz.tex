\section{Anmerkungen und Relevanz}
Dieser Abschnitt enth\"alt Anmerkungen zu dem Originalpaper. Teilweise unterliegen diese pers\"onlichen Eindr\"ucken.\\
%Das Originalpaper setzt sich mit einem Ansatz auseinerander der es erm\"oglichen soll Metamodelle hoch skalierbar und basierend auf JSON zu entwerfen.
Diese Arbeit scheint ein Pionierprojekt zu sein und zentriert sich Inhaltlich um die Evaluation gegen\"uber etablierten Modellen wie ECore und weniger um die Beschreibung des Modells. Es gibt keine Querverweise zu vorarbeiten im Zusammenhang mit MoDiGen, dies unterst\"uzt die Annahme, dass im Rahmen von ProGraMof an der HTWG Konstanz, aktiv an diesem Thema gearbeitet wird.
\\\\
Ein grosses Problem stellen die Internet-Referenzen zu der Projekthomepage von MoDiGen (\url{http://www.modigen.de}) dar. Teilweise sind die Seiten nicht gepflegt und die Repositories f\"ur die Installationsschritte der Software nicht (mehr) Online.\\ 
Des Weiteren wird innerhalb des Originalpapers auf das vollst\"andige Beispiel der JSON-Dateien aus Unterabschnitt \ref{subsec:beispiel} verwiesen. Dieser Online-Verweis ist jedoch nicht mehr aufzufinden.\\
Dies ist jedoch nur Problematisch, da das UML-Diagramm die Abbildung von Referenzen in den JSON-Daten nicht klar beschreibt. Ein vollst\"andiges Beispiel ist daher unerl\"asslich, um ein hinreichendes Verst\"andnis von dem Modell zu bekommen.\\
Das UML-Diagramm wurde unsorgf\"altig entworfen, da es Fehler enth\"alt und einige Sachverhalte besser h\"atten dargestellt werden k\"onnen. Es bleibt unklar, was Scalar/Skalar darstellen soll. Der Wert eines Attributes fehlt in dem Modell g\"anzlich. 
