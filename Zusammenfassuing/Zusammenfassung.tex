%%%%%%%%%%%%%%%% Präambel-ANFANG
% Die folgenden Zeilen zwischen Präambel-ANFANG und Präambel-ENDE bitte nicht ändern.

% Das Latex-Dokument beginnt mit dem \documentclass-Befehl, welcher eine angepasste Version der ACM sig-alternate-Klasse (ewks-latex) referenziert. Die Datei ewks-latex.cls muss sich im gleichen Verzeichnis wie die Zusammenfassung befinden.
\documentclass[ngerman]{ewks-latex}

% Anpassungen an die deutsche Sprache.
\usepackage[ngerman]{babel}

% Angaben zu Zeichensätzen (unter anderem ist Nutzung deutscher Umlaute möglich).
\usepackage[T1]{fontenc}
\usepackage[utf8]{inputenc}

% Entfernung von für den Copyright-Hinweis der Konferenz reserviertem Platz in der ersten Spalte.
\usepackage{etoolbox}
\usepackage{color}

\newcommand{\todo}[2]{\textcolor{red}{\underline{\textbf{TODO [#1]}} \\ #2}}

\makeatletter
\patchcmd{\maketitle}{\@copyrightspace}{}{}{}
\makeatother

% Anzahl der Autoren der Zusammenfassung (in diesem Semester zwei Studenten).
\numberofauthors{2}

%%%%%%%%%%%%%%%% Präambel-ENDE


% Das listings-Paket kann zum Setzen von Quellcode verwendet werden. Es kann auch entfernt werden, falls kein Quellcode gesetzt werden muss.
\usepackage{listings}

% Start des Dokuments.
\begin{document}

% Angabe des *englischen* Titels des Papers.
% Bitte das Tag <TITEL DES ORIGINALS> entsprechend ersetzen.
\title{<TITEL DES ORIGINALS>}

% Als Untertitel sollte darauf hingewiesen werden, dass es sich um eine Zusammenfassung des Papers der jeweiligen Konferenz inkl. Jahrgang handelt.
% Bitte das Tag <JAHRGANG> entsprechend ersetzen.
\subtitle{Zusammenfassung des auf der <KONFERENZNAME> <JAHRGANG> veröffentlichten, gleichnamigen Papers}

% Angaben zu den Autoren der *Zusammenfassung*.
% Bitte die Tags <VORNAMEx>, <NAMEx>, <MATRIKELNRx>, <EMAILx> durch die eigenen Angaben ersetzen.
\author{
	% Erster Autor (i. d. R. der Corresponding Author, der auch die Mail zur Gruppenanmeldung und Papier-Auswahl verfasst hat).
	\alignauthor
	<VORNAME1> <NAME1>\\
	\affaddr{Fachhochschule Dortmund, Fachbereich Informatik}\\
    \affaddr{<MATRIKELNR1>}\\
    \email{<EMAIL1>@stud.fh-dortmund.de}    
	% Zweiter Autor (diese Kommentarzeile nicht entfernen!)
	\alignauthor
	<VORNAME2> <NAME2>\\
	\affaddr{Fachhochschule Dortmund, Fachbereich Informatik}\\
    \affaddr{<MATRIKELNR2>}\\
    \email{<EMAIL2>@stud.fh-dortmund.de}    
}

% Titel setzen. Dieser Befehl sollte an dieser Stelle belassen werden und vor dem eigentlichen Text auftauchen.
\maketitle

\section{Abstract}
Domänen spezifische Modellierungssprachen, wie ECore, ermöglichen die Beschreibung eines Sachverhaltes mit anschließender Codegenerierung. ECore wird in XML beschrieben und speichert Entitäten mit ihren Verbindungen. 

% Das Paper wie auch die Zusammenfassung gliedern sich in mehrere Abschnitte (\section-Befehl). Der erste Abschnitt sollte der Einleitung (Introduction) vorbehalten sein und eine Spalte nicht überschreiten.
\section{Einleitung (Blindtext)}
% Der folgende Befehl generiert Blindtext für die Einleitung und kann entfernt werden.
\cite{latex}

\todo{Sascha}{Vordefiniertes Listing erstellen \\Abstract fertigstellen}
\begin{lstlisting}[float, caption=Beispiel eines Java-Listings, label=lst:example, language=java, basicstyle=\small, tabsize=4, showstringspaces=false, numbers=left, frame=single, keywordstyle=\bfseries]
public class Main {
	public static void main(String[] args) {
		System.out.println("Hallo Welt");
	}
}
\end{lstlisting}

%%%%%%%%%%%%%%%% Epilog-ANFANG
% Die folgenden Zeilen zwischen Epilog-ANFANG und Epilog-ENDE bitte nicht ändern.

% Neue Seite für das Literaturverzeichnis
\newpage

% Der voreingestellte BibTeX-Stil ist abbrv und sollte nicht geändert werden. Er führt dazu, dass die Quellenverweise durchnummeriert werden, was die Regel für wissenschaftliche Veröffentlichungen in den Naturwissenschaften ist.
\bibliographystyle{abbrv}
%%%%%%%%%%%%%%%% Epilog-ENDE

% Verweis auf die BibTeX-Datei. Diese sollte im gleichen Ordner wie die tex-Datei liegen. ACHTUNG: Der Dateiname ist ohne die Endung .bib anzugeben.
\bibliography{literatur}
\end{document}
