\pagebreak
\rule{0.45\textwidth}{0.4pt}
\pagebreak
\rule{0.45\textwidth}{0.4pt}
\pagebreak


\section{Hinweise zu den folgenden Abschnitten}
Nach der Einleitung sollte in den folgenden Abschnitten der Inhalt des Papers vorgestellt werden. Die Einteilung der Abschnitte kann sich dabei am Paper orientieren.

\section{Hinweise zur Formatierung}
Ein paar Beispiele zur Formatierung von Text:
\begin{enumerate}
\item \textit{kursiv}
\item \textbf{fett}
\item \texttt{Code}
\end{enumerate}

\section{Tabellen}
\begin{table} % Einleitung der Tabelle
\centering % �berschrift zentriert
\caption{Ein paar Java-Schl�sselw�rter} % �berschrift
\label{tab:example} % Label f�r sp�tere Referenzierung per \ref
\begin{tabular}{|c|c|} \hline % Einleitung des Tabelleninhalts
Schl�sselwort & Bedeutung \\ \hline % Zeile 1; Spaltentrennung per &
\texttt{abstract} & Abstrakte Klasse/Methode \\ \hline % Zeile 2
\texttt{class} & Einleitung einer Klasse \\ \hline % Zeile 3
\texttt{throw} & Exception werfen \\ \hline % Zeile 4
\end{tabular}
\end{table}

Tabellen lassen sich mit Hilfe der \texttt{table}- und \texttt{tabular}-Umgebungen setzen. Als Beispiel f�r eine konkrete Anwendung siehe Tabelle \ref{tab:example}.

Warum ist meine Tabelle nicht dort, wo ich sie definiert habe? Weil die Vorlage den "`floating"'-Modus von \LaTeX verwendet und Tabellen, Abbildungen und andere Elemente so anordnet, dass wenig Leerzeilen entstehen.

Die Positionierung der Elemente sollte dabei nicht manuell ge�ndert werden, au�er es handelt sich um doppespaltige Tabellen, Abbildungen etc. Ein Beispiel f�r ein doppelspaltiges Element zeigt Abbildung \ref{fig:doppelspaltig}.

Doppelspaltige Elemente werden durch Varianten der einleitenden Befehle mit dem Suffix "`*"' gesetzt:
\begin{itemize}
\item Tabellen: \texttt{begin\{table*\}}
\item Abbildungen (siehe Abschnitt \ref{sec:figures}): \texttt{begin\{figure*\}}
\end{itemize}

\section{Abbildungen}
\label{sec:figures}
Abbildungen werden durch die \texttt{figure}-Umgebung gesetzt. Als Format empfiehlt sich PostScript (ps) oder Encapsulated PostScript (eps).

\begin{figure}
\centering
\epsfig{file=Quadrat.eps,height=1in,width=1in}
\caption{Ein einspaltiges graues Quadrat (eps-Format).}
\end{figure}

\begin{figure*}
\centering
\epsfig{file=Quadrat.eps,height=1in,width=1in}
\caption{Ein doppelspaltiges graues Quadrat (eps-Format).}
\label{fig:doppelspaltig}
\end{figure*}

\section{Verweise}
Mit Hilfe von \LaTeX \ lassen sich eine Reihe von Verweisen auf weiterf�hrende Informationen nutzen.

Die wichtigste "`Verweistechnik"' ist der Literaturverweis. Der Befehl f�r einen Literaturverweis lautet \texttt{\textbackslash cite\{<Bib\TeX-MARKE>\}}. \texttt{<Bib\TeX-MARKE>} ist dabei durch die jeweilige Bib\TeX-Marke zu ersetzen. Bei Bib\TeX \ handelt es sich um ein Programm, mit dem sich Literaturverweise definieren lassen. Es bietet eine Anbindung an \LaTeX. Ein Beispiel f�r eine Bib\TeX-Datei ist "`literatur.bib"', die sich im selben Verzeichnis wie diese \texttt{.tex}-Datei befinden sollte. Der Befehl \texttt{\textbackslash cite\{latex\}} f�hrt dann zu folgender Ausgabe: \cite{latex}.

Die Bib\TeX-Datei "`literatur.bib"' k�nnen Sie entweder mit einem Texteditor anpassen oder Sie nutzen ein Tool wie JabRef\footnote{\texttt{http://jabref.sourceforge.net}}. Fu�noten setzt man �brigens mit \texttt{\textbackslash footnote\{\}}.

\section{Formeln}
Eine gro�e St�rke von \LaTeX \ ist das Setzen von Formeln. Formeln k�nnen \textit{inline} gesetzt werden, d. h. sie erscheinen im Flie�text: $\lim_{n\rightarrow \infty}x=0$.

Formeln werden entweder in zwei \$...\$ eingeschlossen (Kurzform) oder sind von \texttt{\textbackslash begin\{$x$\}}...\texttt{\textbackslash end\{$x$\}} umgeben mit $x\in\left\{ \mathtt{math\textrm{,}\:\mathtt{equation\textrm{,}\: displaymath}}\right\}$ (Langform).

Formeln k�nnen auch in abgesetzter Form erscheinen, undzwar...
\begin{itemize}
\item ... nummeriert (\texttt{equation}-Umgebung):
\begin{equation}
{n+1\choose k} = {n\choose k} + {n \choose k-1}
\end{equation}

\item ... nicht-nummeriert (\texttt{displaymath}-Umgebung):
\begin{displaymath}
|x| = \left\{ \begin{array}{rl}
 -x &\mbox{ falls $x<0$} \\
  x &\mbox{ sonst}
       \end{array} \right.
\end{displaymath}
\end{itemize} 


\section{Quellcode}
Ab und an muss auch Quelltext gesetzt werden. Dies geschieht mit Hilfe des \texttt{listings}-Pakets\footnote{\texttt{http://texdoc.net/texmf-dist/doc/latex/listings/\\listings.pdf}}. Listing \ref{lst:example} zeigt ein Beispiel f�r ein "`floating"'-Listing.

\begin{lstlisting}[float, caption=Beispiel eines Java-Listings, label=lst:example, language=java, basicstyle=\small, tabsize=4, showstringspaces=false, numbers=left, frame=single, keywordstyle=\bfseries]
public class Main {
	public static void main(String[] args) {
		System.out.println("Hallo Welt");
	}
}
\end{lstlisting}

\section{Abschnitt 8 (Blindtext)}
\lipsum[5-10]

\section{Abschnitt 9 (Blindtext)}
\lipsum[11-20]

\section{Abschnitt 10 (Blindtext)}
\lipsum[21-30]

\section{Abschnitt 11 (Blindtext)}
\lipsum[31-32]

\section{Related Work (Blindtext)}
\lipsum[33-37]

\section{Zusammenfassung (Blindtext)}
\lipsum[42-44]

%%%%%%%%%%%%%%%% Epilog-ANFANG
% Die folgenden Zeilen zwischen Epilog-ANFANG und Epilog-ENDE bitte nicht �ndern.